\section{Estructura del informe final}
Se propone la siguiente estructura de informe final contemplado para el proyecto de trabajo de un año, abarcando Trabajo de Título 1 y 2:

\subsection*{Trabajo de Título 1}
\begin{itemize}
	\item \textbf{Resumen general:}Breve descripción introductoria a los contenidos del informe, de no más de 4 párrafos.
	\item \textbf{Resumen ejecutivo:}Sección que contiene en más detalle la información más relevante del proyecto, incluyendo:
	      \begin{itemize}
		      \item \textbf{Objetivos, Alcances y Limitaciones} que hayan sido encontrados y propuestos para el proyecto.
		      \item \textbf{Metodología}, que explica y justifica los métodos utilizados para llevar a cabo este proyecto.
		      \item \textbf{Análisis de resultados} que brevemente entrega un diagnóstico de lo trabajado, y el resultado final del proyecto con respecto al pronóstico inicial.
	      \end{itemize}
	\item \textbf{Planteamiento:} que funciona como una introducción real al proyecto, con los objetivos generales y específicos, alcances y limitaciones, y metodología de trabajo en detalle.
	\item \textbf{Estado actual y marco teórico:} correspondiente a un modelado de la situación inicial del problema planteado por SISEI.
	      % TODO: Agregar el resto de capítulos cuando se termine de armar el cronograma
	\item \textbf{Requerimientos:} corresponde a una sección que documente la toma de requerimientos funcionales y no funcionales del proyecto, junto con todas las acotaciones encontradas y notas que se vean a lo largo de que este proyecto avance.
	\item \textbf{Diseño inicial de la solución:} donde se modela una posible arquitectura inicial del proyecto previo a la implementación. Debido a la naturaleza ágil del proyecto es posible que este modelo inicial no sea reflejado en exactitud por el producto final. Debe, sin embargo, servir como un punto de partida sólido para la solución.
\end{itemize}

\subsection*{Trabajo de Título 2}
\begin{itemize}
	\item \textbf{Desarrollo:}que contempla la documentación de todo el proceso de desarrollo del producto o solución.
	\item \textbf{Implementación:} que contempla la documentación de todo el proceso de la implementación de esta solución en los procesos de SISEI.
	\item \textbf{Pruebas y Validación:} que contempla la documentación de todas las pruebas y validaciones hechas para asegurar la calidad de la solución.
	\item \textbf{Conclusiones y recomendaciones:} Donde se dan los pensamientos finales del resultado del proyecto, así como el desarrollo del mismo, y cómo se debería continuar con esta solución en caso de que así se desee hacer, así también el cómo mantenerla en el tiempo.
\end{itemize}

\subsection*{Secciones universales}
\begin{itemize}
	\item \textbf{Bibliografía:} donde se adjunta todas las referencias bibliográficas utilizadas a lo largo del proyecto.
	\item \textbf{Anexos:} Cualquier recurso adicional que pueda servir para la mejor comprensión del documento.
\end{itemize}

\clearpage
