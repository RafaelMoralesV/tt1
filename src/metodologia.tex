
\section{Metodología}
  \subsection{Agile: SCRUM}
    La naturaleza del proyecto hace que no haya un punto de partida definido y `escrito en piedra`. SISEI como cliente tiene una imagen bien estructurada de lo que quiere según ha leído en algunos blogs, pero no hay un proyecto guía con el que puedan tener la precisión en esta definición, por lo que una parte importante del proyecto será definir con exactitud qué es lo que quiere SISEI como cliente.

    SCRUM normalmente divide el proyecto en iteraciones cortas (llamadas "sprints") con duración típica de 1 a 4 semanas, en donde se define desde un inicio qué trabajo se hará. Debido a que este proyecto contiene un cronograma de actividades (ver tabla \ref{tab:cronograma}), es sencillo de definir estas tareas y controlar el progreso del proyecto.

    Adicionalmente, como existen hitos definidos con plazos de 4 semanas (ver tabla \ref{tab:hitos}), hay un buen argumento para considerarlos puntos críticos para el fin de cada Sprint. Sin embargo, a fin de mantener un rumbo definido con certeza por SISEI como cliente, se tomarán reuniones semanales, donde se revisará progreso y resolverá dudas que se generen, por lo que se sugiere utilizar una suerte de 'mid-sprints' cada dos semanas donde redireccionar el proyecto y revisar su progreso, lo que resultaría en un ciclo de reuniones de progreso y resolución de dudas, acompañado por reuniones de control de estado y ajuste, junto con planificación de nuevas actividades.

    Se planea de esta forma utilizar esta metodología desde el inicio, para mantener una alta agilidad desde la toma de requerimientos hasta la entrega del producto final.

\clearpage
