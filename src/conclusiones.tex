\section{Conclusiones}

Durante este primer espacio del trabajo de título se cruzaron las dos principales partes del proyecto, la toma de requerimientos y la investigación de la solución, que son el cimiento fundamental desde donde se trabajará en la implementación de la solución.

A nivel de introspectiva, definitivamente fue curioso lo sólidamente cimentado que está el stack ELK en este tipo de problemáticas a nivel de mercado, pero este resultado solo me llena más de confianza de que su uso es el correcto para el caso particular de SISEI.

Kibana, por suerte, ofrece dashboards hipotéticos para proyectos similares, según se vio en la sección de progreso realizado, particularmente en la figura \ref{img:ss_dashboard}. Se espera que el proyecto, entonces, logre un espacio similar para el análisis de datos provisto por los logs en el contexto de SISEI. Aún hay decisiones que tomar en cuanto a la transformación y enrriquecimiento de la información que estos logs pueden aportar, pero esto está contemplado como parte de lo que se debería discutir con los StakeHolders conforme avanza el proceso de desarrollo del proyecto.

La expectativa, entonces, es que el proceso de desarrollo esté terminado como última instancia a finales de Septiembre, y que el resto del tiempo se utilice en el desarrollo y análisis de los resultados de este proyecto, y el ámbito académico que esto conlleva.

\clearpage
