
\subsection{Toma de requerimientos especifica}

\begin{enumerate}
  \item \textbf{Para el área de desarrollo: }
    \begin{itemize}
      \item Debe poder acceder a la base de datos de desarrollo y extraer los logs desde la tabla \quotes{general.log\_error}.
      \item Debe extraer los logs desde esta tabla, y transformarlos hacia un formato concreto (por especificar) al momento de almacenarlos en el servidor centralizado.
      \item Eliminar, apartar, o de alguna forma organizar las trazas automáticas (generadas por trabajos de CRON\footnote{Daemon de Linux para ejecutar comandos programados (Vixie Cron) - Man page of CRON}) para que no intervengan o molestan al momento de analizar errores de desarrollo.
        \item Se debe establecer una serie de categorías de Logs para esta zona, como por ejemplo:
          \begin{itemize}
            \item Logs de Acceso.
            \item Logs automáticos CRON.
            \item Logs de desarrollo.
          \end{itemize}
    \end{itemize}
    \item \textbf{Para el área de infraestructura: }
      \begin{itemize}
        \item Debe ser capaz de conectarse y extraer logs en formatos de archivos desde todos los servidores\footnote{Academia, Amanda, MiUtem, Siga y Sigedi}.
        \item Tiene que extraer la información desde dos formatos: texto plano, y GNU Zipped Archive (archivos terminados en .gz).
          \item Debido a la naturaleza de ser de múltiples herramientas distintas, debe ser capaz de transformar todos estos logs a un formato común.
          \item Debería poder obtener logs desde SystemD Journal para el proceso de Unicorn. Normalmente obtenido desde el comando \quotes{journalctl -u gunicorn}.\footnote{Es posible que estos logs estén repetidos en la base de datos de desarrollo.}
      \end{itemize}
\end{enumerate}

\clearpage
