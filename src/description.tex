\section{Descripción del proyecto}

Este proyecto busca ayudar al departamento SISEI de la universidad tecnológica Metropolitana con el manejo de su trazabilidad en las implementaciones y plataformas que han creado a lo largo de los años.

El problema con el que se cuenta hoy en día es una alta variedad de formatos y lugares desde los cuales obtener estos archivos de trazabilidad. Los principales sistemas que tiene la universidad son SIGA, AMANDA, SIGEDI y SISAV, correspondientes a varios de los sistemas informáticos más importantes de la universidad, y que sirven a las diversas Vicerrectorías de la universidad (VRAC, VRAF, VIP y DTT, respectivamente).

Lo anterior ha resultado en problemas al momento de \textit{debuggear} código, pues se necesita buscar logs del problema detectado en máquinas específicas, con credenciales inconsistentes, entre abundantes y diversos tipos de trazas distintas. Para poner énfasis en esto, Sebastián lo compara con `bucear por un mar de logs`.

Adicionalmente, esto ha hecho que la información que es extraíble de estos logs sea perdida. Si SISEI deseara saber qué porcentaje de usuarios utiliza X o Y plataforma en un dispositivo móvil versus un escritorio común, no tiene posibilidad de analizarlo con las herramientas con las que cuenta hoy en día.

Bajo esto, el objetivo último de este proyecto busca solucionar estos dos problemas; conseguir un punto centralizado para estos logs, y crear una plataforma que convierta todos estos datos en información que pueda acompañar a las decisiones futuras del departamento.

\section{Objetivos}

Para el proyecto, se contemplan los siguientes objetivos:

\subsection{Objetivo General}

Modernizar y centralizar en una única plataforma el acceso a los logs y datos de trazabilidad de los sistemas principales de la universidad, con el fin de mejorar la toma de decisiones, habilitando que esta sea basada en datos e información adquirible del producto de este proyecto.

\subsection{Objetivos Específicos}

\begin{itemize}
	\item Crear un modelo de la situación actual, sobre el cual se puedan comparar los cambios propuestos.
	\item Generar un análisis de mercado y estado del arte sobre alternativas y soluciones que se hayan empleado para problemas similares o idénticos, y el cómo fueron estos proyectos llevados a cabo.
	\item Hacer pruebas de concepto y viabilidad de las distintas arquitecturas consideradas.
	\item En base a las pruebas, implementar esta arquitectura elegida sobre los sistemas principales de la Universidad.
	\item Generar y esparcir el conocimiento de uso de esta nueva plataforma sobre los usuarios y stakeholders del proyecto, principalmente el área de desarrollo y mantención de sistemas informáticos.
\end{itemize}

\section{Alcances y limitaciones}

Se plantea a continuación los alcances y limitaciones de este proyecto:

\subsection{Alcances}

\begin{itemize}
	\item La solución a implementar debe contemplar, salvo breves excepciones, los cuatro sistemas informáticos mencionados (SIGA, AMANDA, SIGEDI y SISAV), además de la intranet institucional Mi UTEM.
	\item Deberá mejorar la eficiencia y productividad en cuanto al acceso de estos datos, sea por \textit{debug} o por estudio de los mismos.
	\item El proyecto debe apuntar a generar un producto o solución que ayude en la toma de decisiones informadas por parte de SISEI.
\end{itemize}

\subsection{Limitaciones}
\begin{itemize}
	\item Limitaciones técnicas, en términos de hardware insuficiente, incompatibilidad de software, entre otros, que deben ser superadas durante el proyecto.
	\item Resistencia al cambio por parte del equipo de SISEI u otros, de acuerdo a cómo sean intervenidos los procesos.
	\item Problemas de seguridad, pues los datos que se manejan pueden ser de extrema sensibilidad. Se necesitarán políticas y procedimientos claros para la gestión y acceso de estos datos.
	\item Limitaciones de recursos, principalmente de tiempo. Se debe mantener un alto ojo al alcance de este proyecto conforme avanza.
\end{itemize}

\clearpage
