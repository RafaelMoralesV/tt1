\section{Descripción del proyecto}

Este proyecto busca ayudar al departamento SISEI de la universidad tecnológica Metropolitana con el manejo de su trazabilidad en las implementaciones y plataformas que han creado a lo largo de los años.

El problema con el que se cuenta hoy en día es una alta variedad de formatos y lugares desde los cuales obtener estos archivos de trazabilidad. Los principales sistemas que tiene la universidad son SIGA, AMANDA, SIGEDI y SISAV, correspondientes a varios de los sistemas informáticos más importantes de la universidad, y que sirven a las diversas Vicerrectorías de la universidad (VRAC, VRAF, VIP y DTT, respectivamente).

% TODO: Detallar más el problema.
\lipsum[2]

\section{Objetivos}

Para el proyecto, se contemplan los siguientes objetivos:

\subsection{Objetivo General}

Modernizar y centralizar en una única plataforma el acceso a los logs y datos de trazabilidad de los sistemas principales de la universidad, con el fin de mejorar la toma de decisiones, habilitando que esta sea basada en datos e información adquirible del producto de este proyecto.

\subsection{Objetivos Específicos}

\begin{itemize}
  \item Crear un modelo de la situación actual, sobre el cual se puedan comparar los cambios propuestos.
  \item Generar un análisis de mercado y estado del arte sobre alternativas y soluciones que se hayan empleado para problemas similares o idénticos, y el cómo fueron estos proyectos llevados a cabo.
  \item Hacer pruebas de concepto y viabilidad de las distintas arquitecturas consideradas.
  \item En base a las pruebas, implementar esta arquitectura elegida sobre los sistemas principales de la Universidad.
  \item Generar y esparcir el conocimiento de uso de esta nueva plataforma sobre los usuarios y stakeholders del proyecto, principalmente el área de desarrollo y mantención de sistemas informáticos.
\end{itemize}

\section{Alcances y limitaciones}

Se plantea a continuación los alcances y limitaciones de este proyecto:

\subsection{Alcances}

La solución a implementar debe contemplar, salvo breves excepciones, los cuatro sistemas informáticos mencionados (SIGA, AMANDA, SIGEDI y SISAV), además de la intranet institucional Mi UTEM.

% TODO: Agregar más alcances.

\subsection{Limitaciones}
% TODO: Agregar las limitaciones
\lipsum[5]

% TODO: Cambiar de lugar a su propio archivo, y cambiar de metodología de trabajo.
\section{Metodología Scrum}
\lipsum[10] \\
\lipsum[11] \\
\lipsum[12]

\clearpage

\section{Recursos}
% TODO: Agregar los recursos.
\lipsum[20] \\
\lipsum[21] \\
\lipsum[22] \\
\lipsum[23] \\
\lipsum[24]

\clearpage

% TODO: Detallar la estructura del informe final.
\section{Estructura del informe final}
\clearpage

\section{Plan de trabajo}
\subsection{Cronograma}
% Please add the following required packages to your document preamble:
% \usepackage[table,xcdraw]{xcolor}
% If you use beamer only pass "xcolor=table" option, i.e. \documentclass[xcolor=table]{beamer}
% \usepackage{lscape}
\begin{table}[H]
	\tiny
	\begin{tabular}{llllllll}
		\rowcolor[HTML]{CBCEFB}
		\multicolumn{2}{l}{\cellcolor[HTML]{CBCEFB}\textbf{Actividad}} &
		\multicolumn{2}{l}{\cellcolor[HTML]{CBCEFB}\textbf{Fecha}}     &
		\multicolumn{2}{l}{\cellcolor[HTML]{CBCEFB}\textbf{Duración}}  &
		\multicolumn{2}{l}{\cellcolor[HTML]{CBCEFB}\textbf{Porcentajes}}                                                                                                                        \\
		\rowcolor[HTML]{CBCEFB}
		\textbf{Nro.}                                                  &
		\textbf{Nombre}                                                &
		\textbf{Inicio}                                                &
		\textbf{Término}                                               &
		\textbf{Días}                                                  &
		\textbf{Horas}                                                 &
		\textbf{HH Acum}                                               &
		\textbf{Avance}                                                                                                                                                                         \\
		\rowcolor[HTML]{CBCEFB}
		\multicolumn{8}{c}{\cellcolor[HTML]{CBCEFB}\textbf{Trabajo de Título I}}                                                                                                                \\
		\rowcolor[HTML]{DAE8FC}
		\textbf{E0}                                                    & \textbf{Inicio y Anteproyecto}               & \textbf{}  & \textbf{}  & \textbf{} & \textbf{} & \textbf{} & \textbf{} \\
		0-1                                                            & Objetivos                                    &            &            &           &           &           &           \\
		0-2                                                            & Álcances y Limitaciones                      &            &            &           &           &           &           \\
		0-3                                                            & Recursos                                     &            &            &           &           &           &           \\
		0-4                                                            & Metodología                                  &            &            &           &           &           &           \\
		0-5                                                            & Estructura                                   &            &            &           &           &           &           \\
		0-6                                                            & Plan de Trabajo                              &            &            &           &           &           &           \\
		\rowcolor[HTML]{ECF4FF}
		H0                                                             & Hito 0: Entrega de Anteproyecto              & 27/04/2023 & 27/04/2023 & 1         & 1         &           &           \\
		\rowcolor[HTML]{DAE8FC}
		\textbf{E1}                                                    & \textbf{Propuesta e Investigación}           & \textbf{}  & \textbf{}  & \textbf{} & \textbf{} & \textbf{} & \textbf{} \\
		1-1                                                            & Creación y elaboración de visión de producto &            &            &           &           &           &           \\
		1-2                                                            & Investigación del estado del Arte            &            &            &           &           &           &           \\
		1-3                                                            & Elaboración de diagrama comparativo          &            &            &           &           &           &           \\
		\rowcolor[HTML]{ECF4FF}
		H1                                                             & Hito 1: Entrega de Avance 1                  &            &            &           &           &           &           \\
		\rowcolor[HTML]{DAE8FC}
		\textbf{E2}                                                    & \textbf{Evaluación y Validación}             & \textbf{}  & \textbf{}  & \textbf{} & \textbf{} & \textbf{} & \textbf{}
	\end{tabular}
\end{table}

\subsection{Plan de Hitos}

\clearpage

