\section{Descripción del proyecto}

Este proyecto busca ayudar al departamento SISEI de la universidad tecnológica Metropolitana con el manejo de su trazabilidad en las implementaciones y plataformas que han creado a lo largo de los años.

El problema con el que se cuenta hoy en día es una alta variedad de formatos y lugares desde los cuales obtener estos archivos de trazabilidad. Los principales sistemas que tiene la universidad son SIGA, AMANDA, SIGEDI y SISAV, correspondientes a varios de los sistemas informáticos más importantes de la universidad, y que sirven a las diversas Vicerrectorías de la universidad (VRAC, VRAF, VIP y DTT, respectivamente).

% TODO: Detallar más el problema.
\lipsum[2]

\section{Objetivos}

Para el proyecto, se contemplan los siguientes objetivos:

\subsection{Objetivo General}

Modernizar y centralizar en una única plataforma el acceso a los logs y datos de trazabilidad de los sistemas principales de la universidad, con el fin de mejorar la toma de decisiones, habilitando que esta sea basada en datos e información adquirible del producto de este proyecto.

\subsection{Objetivos Específicos}

\begin{itemize}
  \item Crear un modelo de la situación actual, sobre el cual se puedan comparar los cambios propuestos.
  \item Generar un análisis de mercado y estado del arte sobre alternativas y soluciones que se hayan empleado para problemas similares o idénticos, y el cómo fueron estos proyectos llevados a cabo.
  \item Hacer pruebas de concepto y viabilidad de las distintas arquitecturas consideradas.
  \item En base a las pruebas, implementar esta arquitectura elegida sobre los sistemas principales de la Universidad.
  \item Generar y esparcir el conocimiento de uso de esta nueva plataforma sobre los usuarios y stakeholders del proyecto, principalmente el área de desarrollo y mantención de sistemas informáticos.
\end{itemize}

\section{Alcances y limitaciones}

Se plantea a continuación los alcances y limitaciones de este proyecto:

\subsection{Alcances}

La solución a implementar debe contemplar, salvo breves excepciones, los cuatro sistemas informáticos mencionados (SIGA, AMANDA, SIGEDI y SISAV), además de la intranet institucional Mi UTEM.

% TODO: Agregar más alcances.

\subsection{Limitaciones}
% TODO: Agregar las limitaciones
\lipsum[5]

