% TODO:
% Descripción del sistema propuesto:
%   Explicación de la arquitectura y diseño del sistema de centralización y monitoreo de logs.
%   Descripción de las tecnologías utilizadas y su justificación.
%   Detalles sobre los componentes clave del sistema y su funcionalidad.

\section{Descripción del Sistema propuesto}

Debido a la gran cantidad de paralelismos que existe entre este tipo de proyectos, y un sistema de Business Intelligence (BI) común, podemos empezar a describir el primero en los términos del último.

\insertimage[\label{img:sistema_BI}]{assets/Sistema Propuesto.png}{width=\textwidth}{Sketch de un sistema de Business Intelligence en el contexto del proyecto de centralización de Logs}

El proceso de Business Intelligence consta de 4 componentes simples:

\begin{enumerate}
  \item Una fuente de datos inicial, que sería representado por las bases de datos de Logs de SISEI (AMANDA, SIGA, etc...).
  \item Un proceso de ETL, o de 'Extract, Transform and Load' (Extraer, Transformar y Cargar), en donde los datos de las fuentes son procesados para seguir un formato, calidad y precisión mínimo con el que se puede trabajar.
  \item Data Warehouse, que corresponde al sitio centralizado donde se guardarán estos logs. En el caso del proyecto, podemos pensar en una base de datos orientada a eventos, que permita un acceso eficiente y busqueda de registros.
  \item Con los datos centralizados, es posible extraer información al relacionar eventos y generar tendencias, detectar patrones y buscar anomalias, lo que puede ayudar a detectar problemas y mejorar la seguridad y rendimiento de estos sistemas.
  \item Informes, visualización y alertas pueden ser generadas conforme el sistema sea establecido e implementado paneles de control interactivos que muestren métricas y estadísticas relevantes.
\end{enumerate}

\clearpage
