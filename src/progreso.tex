% TODO: Progreso realizado hasta el momento:
%   Descripción de las tareas y actividades completadas hasta la fecha.
%   Detalles sobre el desarrollo e implementación de los componentes del sistema.
%   Menciona cualquier desafío o contratiempo enfrentado durante el proceso y cómo se han abordado.

\section{Progreso realizado}

Para este segundo hito, según lo marcado en el cronograma inicial (ver tabla \ref{tab:cronograma}), esta primera etapa debería ser principalmente una toma de requerimientos. La viablididad de estas tareas se vio altamente afectada, lamentablemente, por el contexto actual y común de SISEI, por lo que la realización de entrevistas no pudo ser llevado a cabo. Bajo esto, se realizó un cambio en el enfoque del primer avance, desde uno de toma de requerimientos a uno principalmente investigativo en naturaleza, según se ve en el capítulo \ref{section:descripcion}.

Sin embargo, se creó un documento con notas de los requerimientos más básicos que se deben respetar en este proyecto:

\begin{enumerate}
  \item \textbf{Requerimientos Funcionales:} 
    \begin{itemize}
      \item Capacidad de procesar y analizar grandes volúmenes de logs.
      \item Interfaz de usuario amigable, intuitivo.
      \item Seguridad adecuada para los datos almacenados
      \item Recopilación de datos de distintos sistemas
      \item Almacenamiento de datos centralizado
      \item Monitoreo a tiempo real
      \item Alertas y notificaciones en caso de problemas
      \item Informes y estadísticas estilo BI
      \item Escalabilidad para futuras necesidades
      \item Integración plena con los sistemas existentes universitarios
      \item Mantenible a lo largo del tiempo
      \item Debe mantener trazabilidad de uso, quien se loguea, cuando, etc.
        \begin{itemize}
          \item Bajo esta idea, cada usuario debe ser distinto, un usuario comodin 'equipo SISEI' derrota el propósito.
        \end{itemize}
    \end{itemize}
  \item \textbf{Requerimientos No Funcionales:} 
    \begin{itemize}
      \item Debe saber recuperarse en caso de perder conectividad (Cortes de luz, fallos de internet, etc).
      \item Debe ser accedible en la mayor cantidad de disposivos.
      \item Implementación de autenticación y autorización para controlar acceso a la plataforma.
      \item Respaldo y recuperación de datos para garantizar la disponibilidad y evitar la pérdida de información crítica.
    \end{itemize}
\end{enumerate}


\clearpage
