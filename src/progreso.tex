\section{Progreso realizado}

% TODO: Importantes a describir:
% La toma de requerimientos,
% La arquitectura diseñada
% Las maquinas virtuales

El proyecto ha abarcado las primeras dos etapas de progreso, y parte de la tercera, según lo descrito en la tabla \ref{tab:etapas}. Estas actividades contemplan la toma de requerimientos, la investigación inicial, y el desarrollo de la solución.

Se hizo un cambio a la planificación inicial en cuanto al orden de estas etapas, pues pese a que se quería generar una toma de requerimientos, lo primero que terminó realizándose fue una investigación de las tecnologías sugeridas por SISEI, al igual que el estado del arte para proyectos similares hasta donde se entendía la problemática. Parte de la toma de requerimientos base se hizo en esta etapa igualmente, pero lo principal fue la investigación y contraste con soluciones existentes según se habla en la sección \ref{anexos:similitud_bi} de los anexos.

Completada esta etapa que coincidió con el primer hito de Trabajo de título, se tuvo una entrevista exitosa con los principales StakeHolders de este proyecto, encargados del área de desarrollo y el área de infraestructura de SISEI, donde se estableció los requerimientos específicos de esta solución según fueron detallados en la sección \ref{anexos:reqs_especificos} de los anexos. Logrado esto, fue posible establecer una versión madura de la arquitectura de la solución, detallada en la sección \ref{section:descripcion}.

Como último avance, se inició de lleno la etapa de desarrollo con la creación de un ambiente de pruebas con máquinas virtuales corriendo Fedora Linux, similar a como lo hacen los servidores de SISEI, y se instaló Elastic Search y Kibana en una de estas, intentando replicar la arquitectura descrita en la sección \ref{section:descripcion} paso a paso.

\clearpage
