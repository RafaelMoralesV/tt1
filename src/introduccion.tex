% TODO: Introducción:
%   Presentación del proyecto y su importancia en la gestión de logs.
%   Descripción del problema que se aborda y la relevancia del mismo.
%   Objetivos específicos del proyecto.

\section{Introduccion}

En la actualidad, el volumen de datos que se produce en la variedad de organizaciones del mundo, particularmente en los últimos años, ha aumentado de gran forma. Estos registros contienen información relevante sobre el funcionamiento de los diversos sistemas, eventos importantes, incidencias y otros. En el caso de la UTEM, y tal como en múltiples otras instituciones y organizaciones, esta información es desperdiciada al no ser gestionada, ya sea de buena forma o gestionada en general.

El presente informe de avance presenta el progreso del proyecto de centralización y monitoreo de logs, cuyo principal objetivo consta de implementar un sistema de recopilación, almacenamiento y análisis, de manera centralizada, de los logs de los diversos sistemas que posee UTEM como institución. Se espera que este sistema ayude a considerablemente a la gestión de los registros de manera más eficiente, así como facilitar la detección temprana de fallos y problemas que estos presenten.

El problema con el que se cuenta hoy en día es una alta variedad de formatos y lugares desde los cuales obtener estos archivos de trazabilidad. Los principales sistemas que tiene la universidad son SIGA, AMANDA, SIGEDI y SISAV\footnote{En orden, estos son los sistemas para: \quotes{Gestión Académica}, \quotes{Administración y Finanzas}, \quotes{Investigación y Posgrado} y \quotes{VcM, TT y Extensión}.}, correspondientes a varios de los sistemas informáticos más importantes de la universidad, y que sirven a las diversas Vicerrectorías de la universidad (VRAC, VRAF, VIP y VTTE, respectivamente)\footnote{En orden, \quotes{Vicerrectoría Académica}, \quotes{Vicerrectoría de Administración y Finanzas}, \quotes{Vicerrectoría de Investigación y Posgrado}, y \quotes{Vicerrectoría de Transferencia tecnológica y extensión}.}.

% TODO: Respaldar esto en la sección de progreso
En esta estapa inicial de proyecto se ha realizado una investigación sobre mejores prácticas y tecnologías disponibles en el campo de la gestión de logs. Se propone una arquitectura inicial que abarca desde la recolección de logs desde diversas aplicaciones y sistemas hasta su posterior análisis en un único sistema centralizado.

Se espera que, conforme se avance en el proyecto, se encuentren desafíos adicionales, como la normalización y el enriquecimiento de estos logs, así como el diseño de un panel de búsqueda e información que sea realmente útil para los propósitos de SISEI. Bajo esto, el objetivo último de este proyecto busca solucionar estos dos problemas; conseguir un punto centralizado para estos logs, y crear una plataforma que convierta todos estos datos en información que pueda acompañar a las decisiones futuras del departamento.

\clearpage

\section{Objetivos}
\subsection{Objetivo General}

Modernizar y centralizar en una única plataforma el acceso a los logs y datos de trazabilidad de los sistemas principales de la universidad, con el fin de mejorar la toma de decisiones, habilitando que esta sea basada en datos e información adquirible del producto de este proyecto.

\subsection{Objetivos Específicos}

\begin{itemize}
	\item Crear un modelo de la situación actual, sobre el cual se puedan comparar los cambios propuestos.
	\item Generar un análisis de mercado y estado del arte sobre alternativas y soluciones que se hayan empleado para problemas similares o idénticos, y el cómo fueron estos proyectos llevados a cabo.
	\item Hacer pruebas de concepto y viabilidad de las distintas arquitecturas consideradas.
	\item En base a las pruebas, implementar esta arquitectura elegida sobre los sistemas principales de la Universidad.
	\item Generar y esparcir el conocimiento de uso de esta nueva plataforma sobre los usuarios y stakeholders del proyecto, principalmente el área de desarrollo y mantención de sistemas informáticos.
\end{itemize}

\clearpage

\section{Alcances y limitaciones}

Se plantea a continuación los alcances y limitaciones de este proyecto:

\subsection{Alcances}

\begin{itemize}
	\item La solución a implementar debe contemplar, salvo breves excepciones, los cuatro sistemas informáticos mencionados (SIGA, AMANDA, SIGEDI y SISAV), además de la intranet institucional Mi UTEM.
	\item Deberá mejorar la eficiencia y productividad en cuanto al acceso de estos datos, sea por \textit{debug} o por estudio de los mismos.
	\item El proyecto debe apuntar a generar un producto o solución que ayude en la toma de decisiones informadas por parte de SISEI.
\end{itemize}

\subsection{Limitaciones}
\begin{itemize}
	\item Limitaciones técnicas, en términos de hardware insuficiente, incompatibilidad de software, entre otros, que deben ser superadas durante el proyecto.
	\item Resistencia al cambio por parte del equipo de SISEI u otros, de acuerdo a cómo sean intervenidos los procesos.
	\item Problemas de seguridad, pues los datos que se manejan pueden ser de extrema sensibilidad. Se necesitarán políticas y procedimientos claros para la gestión y acceso de estos datos.
	\item Limitaciones de recursos, principalmente de tiempo. Se debe mantener un alto ojo al alcance de este proyecto conforme avanza.
\end{itemize}

\clearpage
