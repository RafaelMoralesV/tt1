% TODO: Introducción:
%   Presentación del proyecto y su importancia en la gestión de logs.
%   Descripción del problema que se aborda y la relevancia del mismo.
%   Objetivos específicos del proyecto.

\section{Introduccion}

En la actualidad, el volumen de datos que se produce en la variedad de organizaciones del mundo, particularmente en los últimos años, ha aumentado de gran forma. Estos registros contienen información relevante sobre el funcionamiento de los diversos sistemas, eventos importantes, incidencias y otros. En el caso de la UTEM, y tal como en múltiples otras instituciones y organizaciones, esta información es desperdiciada al no ser gestionada, ya sea de buena forma o gestionada en general.

El presente informe de avance presenta el progreso del proyecto de centralización y monitoreo de logs, cuyo principal objetivo consta de implementar un sistema de recopilación, almacenamiento y análisis, de manera centralizada, de los logs de los diversos sistemas que posee UTEM como institución. Se espera que este sistema ayude a considerablemente a la gestión de los registros de manera más eficiente, así como facilitar la detección temprana de fallos y problemas que estos presenten.

% TODO: Respaldar esto en la sección de progreso
En esta estapa inicial de proyecto se ha realizado una investigación sobre mejores prácticas y tecnologías disponibles en el campo de la gestión de logs. Se propone una arquitectura inicial que abarca desde la recolección de logs desde diversas aplicaciones y sistemas hasta su posterior análisis en un único sistema centralizado.

Se espera que, conforme se avance en el proyecto, se encuentren desafíos adicionales, como la normalización y el enriquecimiento de estos logs, así como el diseño de un panel de búsqueda e información que sea realmente útil para los propósitos de SISEI.

\clearpage
