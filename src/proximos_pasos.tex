\section{Proximos pasos:}

Para este último hito de entrega, y de acuerdo con lo propuesto en el cronograma inicial (ver tabla \ref{tab:cronograma}), el hito de entrega en de informe de Avance final para trabajo de título I corresponde con parte de la tercera etapa del proyecto (Ver tabla de hitos [\ref{tab:hitos}] y tabla de etapas del proyecto [\ref{tab:etapas}]), correspondiente al inicio de la etapa de desarrollo.

La actividad a completar en este periodo de interludio de vacaciones de invierno es el desarrollo de la solución en un entorno de pruebas. Para esto, se ha creado un espacio con dos máquinas virtuales con Fedora Linux y la instalación de los dos componentes principales del stack ELK.

Los siguientes pasos a seguir serán, entonces, la integración de los siguientes componentes del stack hasta lograr una versión cohesiva y funcional del stack entero en este entorno de máquinas virtuales, y luego la implementación en el entorno de desarrollo para este stack, siguiendo los mismos pasos que se harán durante esta etapa.

Para fines de éxito, se debe tener especial cuidado en la documentación de pasos a seguir, configuraciones a crear, y detalles de cómo funciona la arquitectura conforme se está creando, para evitar que la replicación de esta en el nuevo entorno sea propensa a errores. Con el éxito del salto de entorno de pruebas a entorno de desarrollo, el último paso sería una integración hacia el entorno de producción de SISEI.

El tiempo esperado para tener un propotipo funcional de este stack a nivel de entorno de pruebas local es de un mes aproximadamente, junto con otro mes para que esté totalmente funcionando en ambos entornos de SISEI, y se considere listo el proyecto a nivel de cliente.

\clearpage
