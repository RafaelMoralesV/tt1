\section{Proximos pasos:}

De acuerdo a lo propuesto en el cronograma inicial (ver tabla \ref{tab:cronograme}), la segunda etapa del proyecto está en principio completa, pues el diseño inicial de la solución fue propuesto antes de los requerimientos. Se espera que la toma de requerimientos más detallada afecte a este planteamiento inicial en algún grado, junto con la conexión a las bases de datos con los logs en primera mano.

A día de hoy, ya se posee acceso a la VPN de los servidores de desarrollo de SISEI, y acceso SSH a las máquinas de desarrollo de AMANDA y SIGA. Se necesita un punto de acceso a los motores de bases de datos para continuar.

Las siguientes tareas, entonces, corresponderían a una verificación junto con Sebastián Vega, y otros stakeholders de SISEI, para tener una toma de requerimientos más detallada y robusta, a fin de conseguir cualquier requisito funcional y no funcional que no se haya pensado en un inicio, junto con cualquier restricción que no se haya visto en esta primera etapa y que pueda afectar al primer planteamiento de tecnologías.

El propósito final de esta etapa será, entonces, verificar la viabilidad del stack ELK como solución para esta problemática, junto con la verificación de herramientas adicionales que puedan complementarlo, y ajustar el plan inicial según corresponda.

\clearpage
