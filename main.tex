% Template:     Informe LaTeX
% Documento:    Archivo principal
% Versión:      8.1.6 (12/06/2022)
% Codificación: UTF-8
%
% Autor: Pablo Pizarro R.
%        pablo@ppizarror.com
%
% Manual template: [https://latex.ppizarror.com/informe]
% Licencia MIT:    [https://opensource.org/licenses/MIT]

% CREACIÓN DEL DOCUMENTO
\documentclass[
	spanish, % Idioma: spanish, english, etc.
	letterpaper, oneside
]{article}

% INFORMACIÓN DEL DOCUMENTO
\def\documenttitle {Informe de Avance I}
\def\documentsubtitle {Trabajo de Título I}
\def\documentsubject {Sistema de Centralización y monitoreo de logs}

\def\documentauthor {Rafael Ignacio Morales Venegas}
\def\coursename {Ing. Civil en Computación, Mención informática}
\def\coursecode {21041}

\def\universityname {Universidad Tecnológica Metropolitana}
\def\universityfaculty {Facultad de Ingeniería}
\def\universitydepartment {Departamento de Informática y Computación}
\def\universitydepartmentimage {utem_logo}
\def\universitydepartmentimagecfg {height=1.57cm}
\def\universitylocation {Santiago de Chile}

% INTEGRANTES, PROFESORES Y FECHAS
\def\authortable {
	\begin{tabular}{ll}
		Integrante:
		 & \begin{tabular}[]{l}
			   Rafael Morales Venegas
		   \end{tabular}                                  \\
		Profesor:
		 & \begin{tabular}[t]{l}
			   Mauro Castillo Valdés
		   \end{tabular}                                   \\
		Auxiliares:
		 & \begin{tabular}[t]{l}
			 Sebastián Vega Sánchez 	\\
			 Fabián Coloma Vega
		   \end{tabular}                                  \\
		\multicolumn{2}{l}{Fecha de realización: \today}          \\
		\multicolumn{2}{l}{Fecha de entrega: 22 de Mayo de 2023} \\
		\multicolumn{2}{l}{\universitylocation}
	\end{tabular}
}

% IMPORTACIÓN DEL TEMPLATE
\input{template}

% INICIO DE PÁGINAS
\begin{document}

% PORTADA
\templatePortrait

% CONFIGURACIÓN DE PÁGINA Y ENCABEZADOS
\templatePagecfg

% RESUMEN O ABSTRACT
\begin{abstractd}
	La etapa inicial que se planteó en el cronograma correspondía a una de toma de requerimientos. Esto, sin embargo, no pudo llevarse a cabo según el plan debido a diversos factores de disponibilidad, por lo que se ha hecho un cambio a la etapa abordada en este avance, tomando un enfoque principalmente investigativo y de propuesta base.

	La definición de requerimientos, junto con la identificación concreta de stakeholders y entrevistas que se les hiciera, fue pospuesta al siguiente avance y pendiente a revisión. Sin embargo, y en base a la investigación inicial, se recopiló un conjunto de requerimientos básicos para servir de punto base sobre los que construir, además de concretarse la solución a nivel únicamente tecnológico, junto con alternativas que podrían ser utilizadas en caso de cualquier limitación que no hubiera sido atendida en un inicio.

\end{abstractd}

% TABLA DE CONTENIDOS - ÍNDICE
\templateIndex

% CONFIGURACIONES FINALES
\templateFinalcfg

% ======================= INICIO DEL DOCUMENTO =======================

% TODO: Introducción:
%   Presentación del proyecto y su importancia en la gestión de logs.
%   Descripción del problema que se aborda y la relevancia del mismo.
%   Objetivos específicos del proyecto.

\section{Introduccion}

En la actualidad, el volumen de datos que se produce en la variedad de organizaciones del mundo, particularmente en los últimos años, ha aumentado de gran forma. Estos registros contienen información relevante sobre el funcionamiento de los diversos sistemas, eventos importantes, incidencias y otros. En el caso de la UTEM, y tal como en múltiples otras instituciones y organizaciones, esta información es desperdiciada al no ser gestionada, ya sea de buena forma o gestionada en general.

El presente informe de avance presenta el progreso del proyecto de centralización y monitoreo de logs, cuyo principal objetivo consta de implementar un sistema de recopilación, almacenamiento y análisis, de manera centralizada, de los logs de los diversos sistemas que posee UTEM como institución. Se espera que este sistema ayude a considerablemente a la gestión de los registros de manera más eficiente, así como facilitar la detección temprana de fallos y problemas que estos presenten.

% TODO: Respaldar esto en la sección de progreso
En esta estapa inicial de proyecto se ha realizado una investigación sobre mejores prácticas y tecnologías disponibles en el campo de la gestión de logs. Se propone una arquitectura inicial que abarca desde la recolección de logs desde diversas aplicaciones y sistemas hasta su posterior análisis en un único sistema centralizado.

Se espera que, conforme se avance en el proyecto, se encuentren desafíos adicionales, como la normalización y el enriquecimiento de estos logs, así como el diseño de un panel de búsqueda e información que sea realmente útil para los propósitos de SISEI.

\clearpage

% TODO:
% Descripción del sistema propuesto:
%   Explicación de la arquitectura y diseño del sistema de centralización y monitoreo de logs.
%   Descripción de las tecnologías utilizadas y su justificación.
%   Detalles sobre los componentes clave del sistema y su funcionalidad.

\section{Descripción del Sistema propuesto}

\subsection{Similitudes con Business Intelligence}

Debido a la gran cantidad de paralelismos que existe entre este tipo de proyectos, y un sistema de Business Intelligence (BI) común, podemos empezar a describir el primero en los términos del último.

\insertimage[\label{img:sistema_BI}]{assets/Sistema Propuesto.png}{width=\textwidth}{Sketch de un sistema de Business Intelligence en el contexto del proyecto de centralización de Logs}

El proceso de Business Intelligence consta de 4 componentes simples:

\begin{enumerate}
  \item Una fuente de datos inicial, que sería representado por las bases de datos de Logs de SISEI (AMANDA, SIGA, etc...).
  \item Un proceso de ETL, o de 'Extract, Transform and Load' (Extraer, Transformar y Cargar), en donde los datos de las fuentes son procesados para seguir un formato, calidad y precisión mínimo con el que se puede trabajar.
  \item Data Warehouse, que corresponde al sitio centralizado donde se guardarán estos logs. En el caso del proyecto, podemos pensar en una base de datos orientada a eventos, que permita un acceso eficiente y busqueda de registros.
  \item Con los datos centralizados, es posible extraer información al relacionar eventos y generar tendencias, detectar patrones y buscar anomalias, lo que puede ayudar a detectar problemas y mejorar la seguridad y rendimiento de estos sistemas.
  \item Informes, visualización y alertas pueden ser generadas conforme el sistema sea establecido e implementado paneles de control interactivos que muestren métricas y estadísticas relevantes.
\end{enumerate}

\subsection{Sobre las tecnologías}

La recomendación inicial con la que partió el proyecto se armó en base del stack ELK, que es una combinación de Elasticsearch, Logstash y Kibana, que proporciona una solución integral para la centralización, procesamiento y visualización de logs, incluyendo funcionalidades de seguridad. Todas estas herramientas son creadas por la misma organización, y son dispuestas como código abierto.

Cualquier nueva tecnología a considerar tiene que poder superar la facilidad de integración de herramientas que tiene este stack por si mismo, al igual que ser gratuito de la misma forma que lo es el stack. Finalmente, debe encontrar un beneficio que el stack no provea, por lo que genera pocas opciones a considerar.

Sin embargo, es necesario mantener alternativas, pues es totalmente posible que el stack ELK no cumpla los requerimientos que tenga SISEI, o no sea posible implementarlo por alguna u otra razón.

\subsubsection{Stack ELK}

En un inicio, el stack ELK consistía de las herramientas Elasticsearch, Logstash y Kibana, pero adicionalmente fue agregado Beats, por la misma compañía Elastic. Hoy en día el stack es reconocido como un conjunto en sí mismo, por lo que Elastic lo ofrece de manera gratuita bajo el nombre de producto Elastic Stack. Adicionalmente, Elastic ofrece servicios de almacenamiento, suscripción, soporte y otros de manera pagada.

\begin{itemize}
  \item \textbf{Elasticsearch} es un motor de busqueda y análisis distribuiudo y escalable,. Es el componente central del stack ELK y se utiliza para almacenar y buscar datos a tiempo real, incluyendo logs. Esta herramienta usa estructura de índices invertidos para almacenar y buscar datos rápidamente, organizando los datos en clústeres y nodos para lograr una alta disponibilidad y escalabilidad. Adicionalmente, es posible escalar de manera horizontal según crezca la carga de trabajo.
  \item \textbf{Logstash} es una herramienta de procesamiento y transformación de datos que se utiliza para la ingestión de logs y otros tipos de datos en Elasticsearch. Esta herramienta puede recibir datos de diferentes fuentes, sean archivos de logs, bases de datos, sistemas de mensajes, APIs, etc. Además de extraerlos, puede filtrar, enriquecer, transformar y normalizar estos datos. Puede enrutar estos datos a varios destinos, aunque Elasticsearch es normalmente el más común.
  \item \textbf{Kibana} es una plataforma de visualización y análisis de datos que se integra con Elasticsearch para proporcionar una interfaz amigable y potente. Esta ofrece una amplia gama de opciones para visualizar los datos almacenados, incluyendo gráficos, mapas, tablas, diagramas de dispersión y otros. Kibana permite crear cuadros de mando personalizados e interactivos que combinan múltiples visualizaciones en una sola pantalla.
  \item \textbf{Beats} es el integrante más nuevo del stack ELK, que consta de una colección de agentes ligeros y autónomos, utilizados para la recolección de datos y envío de los mismos hacia plataformas Elasticsearch o Logstash. Es el complemento final que permite recopilar datos de diversas fuentes y enviarlos para su procesamiento y análisis. Los beats más interesantes para el proyecto serían:
  \begin{itemize}
    \item \textbf{Filebeat}, para recolección de logs y archivos de registros de diferentes fuentes. Para el caso, es el menos relevante, pues la gran mayoría de logs de los sistemas de SISEI se encuentran en bases de datos.
    \item \textbf{Metricbeat}, encargado de recopilación y envío de métricas del sistema y servicio. Monitoriza el rendimiento de servidores, aplicaciones y otros servicios, y envía estas métricas directo al stack para su análisis y visualización.
    \item \textbf{Packetbeat} está diseñado para capturar y analizar datos de la red en tiempo real. Examina tráfico de red y extrae información sobre las transacciones y comunicaciones que ocurren en la red.
  \end{itemize}
\end{itemize}

\subsubsection{Alternativas a ELK}

Si bien el stack ELK es una solución popular y ampliamente utilizada para la centralización y análisis de logs, existen algunas consideraciones y escenarios en los que podría no ser la mejor opción o en los que otras herramientas podrían ser más adecuadas. A continuación, mencionaré algunas razones por las cuales podrías considerar alternativas al stack ELK:

\insertimage[\label{img:ES_I}]{assets/Elastic Stack I.png}{width=\textwidth}{Diagrama del uso de Elastic Stack, elaborado por David Taylor. \cite{Taylor_2023}}

\begin{enumerate}
\item \textbf{Requerimientos de rendimiento y escala:} En situaciones en las que se maneje un volumen extremadamente alto de logs o se necesite un procesamiento en tiempo real a gran escala, el stack ELK podría no ser la mejor opción. En estos casos, se podrían explorar soluciones como Apache Kafka, Apache Flink o Apache Spark, que están diseñadas específicamente para procesamiento de datos en tiempo real a gran escala.

\item \textbf{Complejidad y recursos:} La configuración y el mantenimiento del stack ELK pueden ser complejos, especialmente en despliegues a gran escala. Además, se requieren recursos significativos en términos de hardware y almacenamiento. Si existen limitaciones de recursos o se busca una solución más liviana y fácil de administrar, se podrían considerar alternativas más sencillas como Graylog, Fluentd o Splunk.

\item \textbf{Integración con otras herramientas y sistemas:} Si ya se utilizan otras herramientas de monitoreo o gestión de logs en la infraestructura, es importante considerar la compatibilidad y la facilidad de integración entre el stack ELK y esas herramientas. Otras soluciones como Datadog, New Relic o Sumo Logic podrían ofrecer una mejor integración con los sistemas existentes.

\item \textbf{Soporte y asistencia técnica:} Dependiendo de las necesidades y la disponibilidad de recursos internos, es relevante considerar el soporte y la asistencia técnica ofrecidos por las diferentes soluciones. Algunas herramientas comerciales, como Splunk, pueden proporcionar un nivel más alto de soporte y servicios profesionales en comparación con el stack ELK de código abierto.

\item \textbf{Casos de uso específicos:} En caso de tener requisitos específicos o necesidades especializadas en términos de análisis de logs, cumplimiento normativo, seguridad, correlación de eventos, entre otros, podrían existir soluciones especializadas que se adapten mejor. Por ejemplo, herramientas como LogRhythm, QRadar o ArcSight ofrecen capacidades avanzadas de seguridad y análisis de logs. 

\end{enumerate}

\clearpage

\section{Progreso realizado}

% TODO: Importantes a describir:
% La toma de requerimientos,
% La arquitectura diseñada
% Las maquinas virtuales

El proyecto ha abarcado las primeras dos etapas de progreso, y parte de la tercera, según lo descrito en la tabla \ref{tab:etapas}. Estas actividades contemplan la toma de requerimientos, la investigación inicial, y el desarrollo de la solución.

Se hizo un cambio a la planificación inicial en cuanto al orden de estas etapas, pues pese a que se quería generar una toma de requerimientos, lo primero que terminó realizándose fue una investigación de las tecnologías sugeridas por SISEI, al igual que el estado del arte para proyectos similares hasta donde se entendía la problemática. Parte de la toma de requerimientos base se hizo en esta etapa igualmente, pero lo principal fue la investigación y contraste con soluciones existentes según se habla en la sección \ref{anexos:similitud_bi} de los anexos.

Completada esta etapa que coincidió con el primer hito de Trabajo de título, se tuvo una entrevista exitosa con los principales StakeHolders de este proyecto, encargados del área de desarrollo y el área de infraestructura de SISEI, donde se estableció los requerimientos específicos de esta solución según fueron detallados en la sección \ref{anexos:reqs_especificos} de los anexos. Logrado esto, fue posible establecer una versión madura de la arquitectura de la solución, detallada en la sección \ref{section:descripcion}.

Como último avance, se inició de lleno la etapa de desarrollo con la creación de un ambiente de pruebas con máquinas virtuales corriendo Fedora Linux, similar a como lo hacen los servidores de SISEI, y se instaló Elastic Search y Kibana en una de estas, intentando replicar la arquitectura descrita en la sección \ref{section:descripcion} paso a paso.

En las siguientes imágenes se provee las pantallas que ofrece Kibana por defecto al momento de instalación como \quotes{Home} (Ver la figura \ref{img:ss_home}), al igual que un ejemplo de Dashboard provisto para un proyecto similar de centralización y monitoreo de variados Logs (Principalmente logs de Acceso a algún servicio hipotético, ver imagen \ref{img:ss_dashboard}).

\insertimage[\label{img:ss_home}]{assets/screenshots/home.png}{width=\textwidth}{Pantalla principal de KIBANA.}

\insertimage[\label{img:ss_dashboard}]{assets/screenshots/dashboard.png}{width=\textwidth}{Ejemplo de Dashboard provisto por KIBANA para gestión y análisis de Logs.}

\clearpage


% TODO: Planificación y cronograma:
%   Una tabla o gráfico que muestre las actividades planificadas y su estado actual.
%   Fechas importantes y hitos alcanzados.
%   Posibles ajustes en el cronograma original y justificación, si los hubiera.

\begin{landscape}
	\section{Plan de trabajo}
	\subsection{Cronograma}
	% Please add the following required packages to your document preamble:
% \usepackage{booktabs}
% \usepackage{longtable}
% Note: It may be necessary to compile the document several times to get a multi-page table to line up properly
\begin{longtable}[c]{@{}llllllll@{}}
	\toprule
	\multicolumn{2}{l}{\textbf{Actividad}} & \multicolumn{2}{l}{\textbf{Fecha}}              & \multicolumn{2}{l}{\textbf{Duración}} & \multicolumn{2}{l}{\textbf{Porcentajes}}                                                                       \\* \midrule
	\textbf{\#}                            & \textbf{Nombre}                                 & \textbf{Inicio}                       & \textbf{Término}                         & \textbf{Días} & \textbf{Horas} & \textbf{HH Acum} & \textbf{Avance} \\* \midrule
	\endhead
	%
	\bottomrule
	\endfoot
	%
	\endlastfoot
	%
	\multicolumn{8}{c}{}                                                                                                                                                                                                                              \\* \midrule
	H0                                     & Inicio del proyecto                             & \multicolumn{2}{l}{12/4/2023}         &                                          &               &                &                                    \\
	1                                      & Objetivos                                       &                                       &                                          &               &                &                  &                 \\
	2                                      & Alcances y Limitaciones                         &                                       &                                          &               &                &                  &                 \\
	3                                      & Recursos                                        &                                       &                                          &               &                &                  &                 \\
	4                                      & Metodología                                     &                                       &                                          &               &                &                  &                 \\
	5                                      & Estructura                                      &                                       &                                          &               &                &                  &                 \\
	6                                      & Plan de Trabajo                                 &                                       &                                          &               &                &                  &                 \\
	H1                                     & Entrega de Anteproyecto                         & \multicolumn{2}{l}{24/04/2023}        &                                          &               &                &                                    \\
	7                                      & Identificación de Stakeholders                  &                                       &                                          &               &                &                  &                 \\
	8                                      & Entrevista con stakeholders                     &                                       &                                          &               &                &                  &                 \\
	9                                      & Definición de requerimientos de datos           &                                       &                                          &               &                &                  &                 \\
	10                                     & Definición de requerimientos de seguridad       &                                       &                                          &               &                &                  &                 \\
	E1                                     & Análisis y levantamiento de requerimientos      &                                       &                                          &               &                &                  &                 \\
	H2                                     & Entrega informe de Avance I                     & \multicolumn{2}{l}{22/05/2023}        &                                          &               &                &                                    \\
	11                                     & Estado del arte                                 &                                       &                                          &               &                &                  &                 \\
	12                                     & Modelado de la arquitectura                     &                                       &                                          &               &                &                  &                 \\
	13                                     & Definición de componentes de la estructura      &                                       &                                          &               &                &                  &                 \\
	E2                                     & Diseño inicial de la solución                   &                                       &                                          &               &                &                  &                 \\
	\textit{H3}                            & \textit{Entrega informe de Avance II}           & \multicolumn{2}{l}{25/06/2023}        &                                          &               &                &                                    \\
	14                                     & Creación de un ambiente de prueba               &                                       &                                          &               &                &                  &                 \\
	15                                     & Integración de la arquitectura planificada      &                                       &                                          &               &                &                  &                 \\
	H4                                     & Entrega informe final de Avance                 & \multicolumn{2}{l}{10/7/2023}         &                                          &               &                &                                    \\* \midrule
	\multicolumn{8}{c}{Trabajo de Título II}                                                                                                                                                                                                          \\* \midrule
	16                                     & Pruebas unitarias                               &                                       &                                          &               &                &                  &                 \\
	E3                                     & Desarrollo                                      &                                       &                                          &               &                &                  &                 \\
	H5                                     & Entrega informe de Avance I                     & \multicolumn{2}{l}{Septiembre 2023}   &                                          &               &                &                                    \\
	17                                     & Integración de sistema a procesos existentes    &                                       &                                          &               &                &                  &                 \\
	E4                                     & Implementación                                  &                                       &                                          &               &                &                  &                 \\
	18                                     & Pruebas funcionales                             &                                       &                                          &               &                &                  &                 \\
	19                                     & Pruebas de carga                                &                                       &                                          &               &                &                  &                 \\
	20                                     & Pruebas de seguridad                            &                                       &                                          &               &                &                  &                 \\
	E5                                     & Pruebas y Validación                            &                                       &                                          &               &                &                  &                 \\
	H6                                     & Entrega informe de Avance II                    & \multicolumn{2}{l}{Octubre 2023}      &                                          &               &                &                                    \\
	21                                     & Entrega de producto / MVP a usuarios            &                                       &                                          &               &                &                  &                 \\
	22                                     & Capacitación de usuarios                        &                                       &                                          &               &                &                  &                 \\
	23                                     & Evaluación de procesos, decisiones y resultados &                                       &                                          &               &                &                  &                 \\
	E6                                     & Cierre del proyecto                             &                                       &                                          &               &                &                  &                 \\
	H7                                     & Entrega Informe final y Resumen Ejecutivo       & \multicolumn{2}{l}{Noviembre 2023}    &                                          &               &                &                                    \\
	H8                                     & Examen de Título                                & \multicolumn{2}{l}{Noviembre 2023}    &                                          &               &                &                                    \\* \bottomrule
	\caption{Cronograma General del Proyecto}
	\label{tab:cronograma}                                                                                                                                                                                                                            \\
\end{longtable}

	\clearpage
\end{landscape}
\subsection{Plan de Hitos}
% TODO: Crear una tabla de hitos.

\subsection{Etapas del Proyecto}
\begin{table}[H]
	\centering
	\begin{tabular}{@{}lll@{}}
		\toprule
		\textbf{Descripción de la Etapa}           & \textbf{Fecha} & \textbf{Avance} \\ \midrule
		Análisis y levantamiento de requerimientos &                &                 \\
		Diseño inicial de la solución              &                &                 \\
		Desarrollo                                 &                &                 \\
		Implementación                             &                &                 \\
		Pruebas y Validación                       &                &                 \\
		Cierre del proyecto                        &                &                 \\ \bottomrule
	\end{tabular}
	\caption{Etapas del Proyecto}
	\label{tab:etapas}
\end{table}

\clearpage

\section{Proximos pasos:}

Para este último hito de entrega, y de acuerdo con lo propuesto en el cronograma inicial (ver tabla \ref{tab:cronograma}), el hito de entrega en de informe de Avance final para trabajo de título I corresponde con parte de la tercera etapa del proyecto (Ver tabla de hitos [\ref{tab:hitos}] y tabla de etapas del proyecto [\ref{tab:etapas}]), correspondiente al inicio de la etapa de desarrollo.

La actividad a completar en este periodo de interludio de vacaciones de invierno es el desarrollo de la solución en un entorno de pruebas. Para esto, se ha creado un espacio con dos máquinas virtuales con Fedora Linux y la instalación de los dos componentes principales del stack ELK.

Los siguientes pasos a seguir serán, entonces, la integración de los siguientes componentes del stack hasta lograr una versión cohesiva y funcional del stack entero en este entorno de máquinas virtuales, y luego la implementación en el entorno de desarrollo para este stack, siguiendo los mismos pasos que se harán durante esta etapa.

Para fines de éxito, se debe tener especial cuidado en la documentación de pasos a seguir, configuraciones a crear, y detalles de cómo funciona la arquitectura conforme se está creando, para evitar que la replicación de esta en el nuevo entorno sea propensa a errores. Con el éxito del salto de entorno de pruebas a entorno de desarrollo, el último paso sería una integración hacia el entorno de producción de SISEI.

El tiempo esperado para tener un propotipo funcional de este stack a nivel de entorno de pruebas local es de un mes aproximadamente, junto con otro mes para que esté totalmente funcionando en ambos entornos de SISEI, y se considere listo el proyecto a nivel de cliente.

\clearpage

\section{Conclusiones}
% TODO: Actualizar conclusiones de acuerdo al avance actual.

El avance hasta ahora a sido un poco más lento de lo que me hubiera gustado, pero en retrospectiva, tiene más sentido que haya tomado este camino inicial por lo dificultoso que es generar una toma de requerimiento con stakeholders con poco tiempo disponible. En ese sentido, cambiar el enfoque del avance fue vital para un buen progreso.

Ya con una idea generalizada de la arquitectura que se va a seguir, y las alternativas disponibles, solamente se necesita concretar la definición real del problema, a un nivel de entendimiento de acorde al de los integrantes de SISEI, para poder empezar a resolverlo correctamente.

Pienso que esta siguiente etapa será decisiva en el resto del proyecto, pues definirá en concreto la forma que tomará la solución. Una parte significativa que se tiene de esta es que no solamente será un stack tecnológico el que hará que este proyecto sea un éxito, sino una definición de protocolos de uso con respecto a los logs. En este sentido, será la afrontación al problema real con la conexión a la base de datos la que hará visible la necesidad y forma que tomarán estos protocolos.

\clearpage



% Referencias que no fueron citadas.
\nocite{alberdi_2020}
\nocite{sharif_2022}
\nocite{schwaber2020scrum}
\nocite{Anonymous}

\bibliography{library}

% FIN DEL DOCUMENTO
\end{document}
