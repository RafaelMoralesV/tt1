% Template:     Informe LaTeX
% Documento:    Archivo principal
% Versión:      8.1.6 (12/06/2022)
% Codificación: UTF-8
%
% Autor: Pablo Pizarro R.
%        pablo@ppizarror.com
%
% Manual template: [https://latex.ppizarror.com/informe]
% Licencia MIT:    [https://opensource.org/licenses/MIT]

% CREACIÓN DEL DOCUMENTO
\documentclass[
	spanish, % Idioma: spanish, english, etc.
	letterpaper, oneside
]{article}

% INFORMACIÓN DEL DOCUMENTO
\def\documenttitle {Trabajo de Título}
\def\documentsubtitle {Informe de Propuesta}
\def\documentsubject {Sistema de Centralización y monitoreo de logs}

\def\documentauthor {Rafael Ignacio Morales Venegas}
\def\coursename {Ing. Civil en Computación, Mención informática}
\def\coursecode {21041}

\def\universityname {Universidad Tecnológica Metropolitana}
\def\universityfaculty {Facultad de Ingeniería}
\def\universitydepartment {Departamento de Informática}
\def\universitydepartmentimage {utem_logo}
\def\universitydepartmentimagecfg {height=1.57cm}
\def\universitylocation {Santiago de Chile}

% INTEGRANTES, PROFESORES Y FECHAS
\def\authortable {
	\begin{tabular}{ll}
		Integrantes:
		 & \begin{tabular}[]{l}
			   Rafael Morales Venegas
		   \end{tabular}                         \\
		Profesor:
		 & \begin{tabular}[t]{l}
			   Mauro Castillo Valdes
		   \end{tabular}                          \\
		Auxiliar:
		 & \begin{tabular}[t]{l}
			   Sebastián Vega Sánchez
		   \end{tabular}                         \\
		\multicolumn{2}{l}{Fecha de realización: \today} \\
		\multicolumn{2}{l}{Fecha de entrega: \today}     \\
		\multicolumn{2}{l}{\universitylocation}
	\end{tabular}
}

% IMPORTACIÓN DEL TEMPLATE
\input{template}

% INICIO DE PÁGINAS
\begin{document}

% PORTADA
\templatePortrait

% CONFIGURACIÓN DE PÁGINA Y ENCABEZADOS
\templatePagecfg

% RESUMEN O ABSTRACT
\begin{abstractd}
	El proyecto propuesto busca centralizar y monitorear los datos generados por los diversos servicios de la universidad a través de un sistema informático eficiente. El objetivo es proporcionar una plataforma sólida y segura para la gestión y supervisión de los registros de actividad de los sistemas y servicios de la universidad, mejorando así la eficacia y la eficiencia de los procesos de gestión de datos. La implementación de esta solución permitirá a la universidad tomar decisiones informadas en tiempo real y garantizar la continuidad de los servicios.

	Debido a la naturaleza de primera implementación de este proyecto, una parte importante de esta implementación será investigativa, donde se buscará herramientas y soluciones ya existentes que puedan ser traídas a producción por la utem, o verificar la viabilidad de implementaciones propias si no existe una alternativa de mercado que satisfazca los requerimientos planteados.

	Como metolodología de ingeniería de software se planea utilizar el framework de SCRUM, adaptado a un desarrollo individual, con los roles de scrum master tomados por el profesor Mauro Castillo, y Product Owner los señores Sebastián Vega y Segundo Fuentes, en representación de SISEI.

	El entregable final planteado para el proyecto es una plataforma web (o similar, que cumpla con la característica de ser revisable en múltiples dispositivos) que reciba estos logs, los unifique en un formato universal para la universidad, y permita la revisión constante de estos por parte del equipo de SISEI.

	\subsection*{Palabras claves}
	\begin{itemize}
		\item Log
		\item Monitoreo de datos
		\item Centralización
		\item Plataforma unificada
	\end{itemize}
\end{abstractd}

% TABLA DE CONTENIDOS - ÍNDICE
\templateIndex

% CONFIGURACIONES FINALES
\templateFinalcfg

% ======================= INICIO DEL DOCUMENTO =======================

\section{Descripción del proyecto}

Este proyecto busca ayudar al departamento SISEI de la universidad tecnológica Metropolitana con el manejo de su trazabilidad en las implementaciones y plataformas que han creado a lo largo de los años.

El problema con el que se cuenta hoy en día es una alta variedad de formatos y lugares desde los cuales obtener estos archivos de trazabilidad,

\lipsum[2]

\section{Objetivos}
\subsection{Objetivo General}
\lipsum[1]

\subsection{Objetivos Específicos}

\section{Alcances y limitaciones}
\lipsum[3]
\subsection{Alcances}
\lipsum[4]
\subsection{Limitaciones}
\lipsum[5]
\section{Metodología Scrum}

\lipsum[10] \\
\lipsum[11] \\
\lipsum[12]

\clearpage

\section{Recursos}

\lipsum[20] \\
\lipsum[21] \\
\lipsum[22] \\
\lipsum[23] \\
\lipsum[24]

\clearpage

\section{Estructura del informe final}
\clearpage

\section{Plan de trabajo}
\lipsum[34] \\
\lipsum[35]

\subsection{Cronograma}
\subsection{Plan de Hitos}

\clearpage



% TODO: Describir la metodología del proyecto.
\section{Metodología Scrum}
\lipsum[10] \\
\lipsum[11] \\
\lipsum[12]

\clearpage

\section{Recursos}
  \subsection{Hardware}
    \begin{itemize}
      \item Servidores y espacios de almacenamiento para Logs
      \item Dispositivos de acceso para el producto final
    \end{itemize}
  \subsection{Software}
    \begin{itemize}
      \item Bases de datos para el almacenamiento de logs
      \item Sist. operativo UNIX, usado por los servidores y el desarrollador.
      \item Software de diagramas, de ofimática y otros.
      \item Ambientes de desarrollo en caso de ser necesario (Editores de texto, compiladores o intérpretes, etc).
      \item Herramientas de gestión, centralización y monitoreo de Logs según sea definido a lo largo del proyecto.
    \end{itemize}

\section{Estructura del informe final}
Se propone la siguiente estructura de informe final contemplado para el proyecto de trabajo de un año, abarcando Trabajo de Título 1 y 2:

\subsection*{Trabajo de Título 1}
\begin{itemize}
	\item \textbf{Resumen general:}Breve descripción introductoria a los contenidos del informe, de no más de 4 párrafos.
	\item \textbf{Resumen ejecutivo:}Sección que contiene en más detalle la información más relevante del proyecto, incluyendo:
	      \begin{itemize}
		      \item \textbf{Objetivos, Alcances y Limitaciones} que hayan sido encontrados y propuestos para el proyecto.
		      \item \textbf{Metodología}, que explica y justifica los métodos utilizados para llevar a cabo este proyecto.
		      \item \textbf{Análisis de resultados} que brevemente entrega un diagnóstico de lo trabajado, y el resultado final del proyecto con respecto al pronóstico inicial.
	      \end{itemize}
	\item \textbf{Planteamiento:} que funciona como una introducción real al proyecto, con los objetivos generales y específicos, alcances y limitaciones, y metodología de trabajo en detalle.
	\item \textbf{Estado actual y marco teórico:} correspondiente a un modelado de la situación inicial del problema planteado por SISEI.
	\item \textbf{Requerimientos:} corresponde a una sección que documente la toma de requerimientos funcionales y no funcionales del proyecto, junto con todas las acotaciones encontradas y notas que se vean a lo largo de que este proyecto avance.
	\item \textbf{Diseño inicial de la solución:} donde se modela una posible arquitectura inicial del proyecto previo a la implementación. Debido a la naturaleza ágil del proyecto es posible que este modelo inicial no sea reflejado en exactitud por el producto final. Debe, sin embargo, servir como un punto de partida sólido para la solución.
\end{itemize}

\subsection*{Trabajo de Título 2}
\begin{itemize}
	\item \textbf{Desarrollo:}que contempla la documentación de todo el proceso de desarrollo del producto o solución.
	\item \textbf{Implementación:} que contempla la documentación de todo el proceso de la implementación de esta solución en los procesos de SISEI.
	\item \textbf{Pruebas y Validación:} que contempla la documentación de todas las pruebas y validaciones hechas para asegurar la calidad de la solución.
	\item \textbf{Conclusiones y recomendaciones:} Donde se dan los pensamientos finales del resultado del proyecto, así como el desarrollo del mismo, y cómo se debería continuar con esta solución en caso de que así se desee hacer, así también el cómo mantenerla en el tiempo.
\end{itemize}

\subsection*{Secciones universales}
\begin{itemize}
	\item \textbf{Bibliografía:} donde se adjunta todas las referencias bibliográficas utilizadas a lo largo del proyecto.
	\item \textbf{Anexos:} Cualquier recurso adicional que pueda servir para la mejor comprensión del documento.
\end{itemize}

\clearpage


% TODO: Planificación y cronograma:
%   Una tabla o gráfico que muestre las actividades planificadas y su estado actual.
%   Fechas importantes y hitos alcanzados.
%   Posibles ajustes en el cronograma original y justificación, si los hubiera.

\begin{landscape}
	\section{Plan de trabajo}
	\subsection{Cronograma}
	% Please add the following required packages to your document preamble:
% \usepackage{booktabs}
% \usepackage{longtable}
% Note: It may be necessary to compile the document several times to get a multi-page table to line up properly
\begin{longtable}[c]{@{}llllllll@{}}
	\toprule
	\multicolumn{2}{l}{\textbf{Actividad}} & \multicolumn{2}{l}{\textbf{Fecha}}              & \multicolumn{2}{l}{\textbf{Duración}} & \multicolumn{2}{l}{\textbf{Porcentajes}}                                                                       \\* \midrule
	\textbf{\#}                            & \textbf{Nombre}                                 & \textbf{Inicio}                       & \textbf{Término}                         & \textbf{Días} & \textbf{Horas} & \textbf{HH Acum} & \textbf{Avance} \\* \midrule
	\endhead
	%
	\bottomrule
	\endfoot
	%
	\endlastfoot
	%
	\multicolumn{8}{c}{}                                                                                                                                                                                                                              \\* \midrule
	H0                                     & Inicio del proyecto                             & \multicolumn{2}{l}{12/4/2023}         &                                          &               &                &                                    \\
	1                                      & Objetivos                                       &                                       &                                          &               &                &                  &                 \\
	2                                      & Alcances y Limitaciones                         &                                       &                                          &               &                &                  &                 \\
	3                                      & Recursos                                        &                                       &                                          &               &                &                  &                 \\
	4                                      & Metodología                                     &                                       &                                          &               &                &                  &                 \\
	5                                      & Estructura                                      &                                       &                                          &               &                &                  &                 \\
	6                                      & Plan de Trabajo                                 &                                       &                                          &               &                &                  &                 \\
	H1                                     & Entrega de Anteproyecto                         & \multicolumn{2}{l}{24/04/2023}        &                                          &               &                &                                    \\
	7                                      & Identificación de Stakeholders                  &                                       &                                          &               &                &                  &                 \\
	8                                      & Entrevista con stakeholders                     &                                       &                                          &               &                &                  &                 \\
	9                                      & Definición de requerimientos de datos           &                                       &                                          &               &                &                  &                 \\
	10                                     & Definición de requerimientos de seguridad       &                                       &                                          &               &                &                  &                 \\
	E1                                     & Análisis y levantamiento de requerimientos      &                                       &                                          &               &                &                  &                 \\
	H2                                     & Entrega informe de Avance I                     & \multicolumn{2}{l}{22/05/2023}        &                                          &               &                &                                    \\
	11                                     & Estado del arte                                 &                                       &                                          &               &                &                  &                 \\
	12                                     & Modelado de la arquitectura                     &                                       &                                          &               &                &                  &                 \\
	13                                     & Definición de componentes de la estructura      &                                       &                                          &               &                &                  &                 \\
	E2                                     & Diseño inicial de la solución                   &                                       &                                          &               &                &                  &                 \\
	\textit{H3}                            & \textit{Entrega informe de Avance II}           & \multicolumn{2}{l}{25/06/2023}        &                                          &               &                &                                    \\
	14                                     & Creación de un ambiente de prueba               &                                       &                                          &               &                &                  &                 \\
	15                                     & Integración de la arquitectura planificada      &                                       &                                          &               &                &                  &                 \\
	H4                                     & Entrega informe final de Avance                 & \multicolumn{2}{l}{10/7/2023}         &                                          &               &                &                                    \\* \midrule
	\multicolumn{8}{c}{Trabajo de Título II}                                                                                                                                                                                                          \\* \midrule
	16                                     & Pruebas unitarias                               &                                       &                                          &               &                &                  &                 \\
	E3                                     & Desarrollo                                      &                                       &                                          &               &                &                  &                 \\
	H5                                     & Entrega informe de Avance I                     & \multicolumn{2}{l}{Septiembre 2023}   &                                          &               &                &                                    \\
	17                                     & Integración de sistema a procesos existentes    &                                       &                                          &               &                &                  &                 \\
	E4                                     & Implementación                                  &                                       &                                          &               &                &                  &                 \\
	18                                     & Pruebas funcionales                             &                                       &                                          &               &                &                  &                 \\
	19                                     & Pruebas de carga                                &                                       &                                          &               &                &                  &                 \\
	20                                     & Pruebas de seguridad                            &                                       &                                          &               &                &                  &                 \\
	E5                                     & Pruebas y Validación                            &                                       &                                          &               &                &                  &                 \\
	H6                                     & Entrega informe de Avance II                    & \multicolumn{2}{l}{Octubre 2023}      &                                          &               &                &                                    \\
	21                                     & Entrega de producto / MVP a usuarios            &                                       &                                          &               &                &                  &                 \\
	22                                     & Capacitación de usuarios                        &                                       &                                          &               &                &                  &                 \\
	23                                     & Evaluación de procesos, decisiones y resultados &                                       &                                          &               &                &                  &                 \\
	E6                                     & Cierre del proyecto                             &                                       &                                          &               &                &                  &                 \\
	H7                                     & Entrega Informe final y Resumen Ejecutivo       & \multicolumn{2}{l}{Noviembre 2023}    &                                          &               &                &                                    \\
	H8                                     & Examen de Título                                & \multicolumn{2}{l}{Noviembre 2023}    &                                          &               &                &                                    \\* \bottomrule
	\caption{Cronograma General del Proyecto}
	\label{tab:cronograma}                                                                                                                                                                                                                            \\
\end{longtable}

	\clearpage
\end{landscape}
\subsection{Plan de Hitos}
% TODO: Crear una tabla de hitos.

\subsection{Etapas del Proyecto}
\begin{table}[H]
	\centering
	\begin{tabular}{@{}lll@{}}
		\toprule
		\textbf{Descripción de la Etapa}           & \textbf{Fecha} & \textbf{Avance} \\ \midrule
		Análisis y levantamiento de requerimientos &                &                 \\
		Diseño inicial de la solución              &                &                 \\
		Desarrollo                                 &                &                 \\
		Implementación                             &                &                 \\
		Pruebas y Validación                       &                &                 \\
		Cierre del proyecto                        &                &                 \\ \bottomrule
	\end{tabular}
	\caption{Etapas del Proyecto}
	\label{tab:etapas}
\end{table}

\clearpage


% FIN DEL DOCUMENTO
\end{document}
